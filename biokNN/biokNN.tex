% !TeX root = RJwrapper.tex
\title{biokNN: A bi-objective imputation method for multilevel data in
R}
\author{by Maximiliano Cubillos, Jesper N. Wulff, and Sanne Wøhlk}

\maketitle

\abstract{%
An abstract of less than 150 words.
}

\begin{Schunk}
\begin{Soutput}
#> 
#> Attaching package: 'dplyr'
\end{Soutput}
\begin{Soutput}
#> The following objects are masked from 'package:stats':
#> 
#>     filter, lag
\end{Soutput}
\begin{Soutput}
#> The following objects are masked from 'package:base':
#> 
#>     intersect, setdiff, setequal, union
\end{Soutput}
\begin{Soutput}
#> 
#> Attaching package: 'mice'
\end{Soutput}
\begin{Soutput}
#> The following objects are masked from 'package:base':
#> 
#>     cbind, rbind
\end{Soutput}
\end{Schunk}

\hypertarget{introduction}{%
\subsection{Introduction}\label{introduction}}

Introductory section which may include references in parentheses
\citep{R}, or cite a reference such as \citet{R} in the text.

\hypertarget{overview-of-bioknn}{%
\subsection{Overview of biokNN}\label{overview-of-bioknn}}

Functions. Most important: biokNN.impute. Explain parameters.

\hypertarget{example-with-simulated-data}{%
\subsection{Example with simulated
data}\label{example-with-simulated-data}}

\begin{Schunk}
\begin{Sinput}
df <- create.multilevel(nClass = 25, nVars = 1, classMean = 10, classSD = 0, 
                        beta0 = 0, tau0 = 1, beta = c(1), tau = c(1), sigma2 = 1)
head(df)
\end{Sinput}
\begin{Soutput}
#>   clust          y        X.df
#> 1     1 -1.8456025 -0.69676039
#> 2     1  5.3353049  1.91506005
#> 3     1 -1.4989518 -0.08173864
#> 4     1  1.5074744  0.37650360
#> 5     1  0.8423699  1.12788195
#> 6     1 -1.2262847 -0.08569532
\end{Soutput}
\end{Schunk}

\hypertarget{data-structure}{%
\subsection{Data structure}\label{data-structure}}

\hypertarget{recommendations-in-using-bioknn-package}{%
\subsection{Recommendations in using biokNN
package}\label{recommendations-in-using-bioknn-package}}

\hypertarget{section-title-in-sentence-case}{%
\subsection{Section title in sentence
case}\label{section-title-in-sentence-case}}

\begin{Schunk}
\begin{Sinput}
x <- 1:10
\end{Sinput}
\end{Schunk}

\hypertarget{summary}{%
\subsection{Summary}\label{summary}}

This file is only a basic article template. For full details of
\emph{The R Journal} style and information on how to prepare your
article for submission, see the
\href{https://journal.r-project.org/share/author-guide.pdf}{Instructions
for Authors}.

\hypertarget{about-this-format-and-the-r-journal-requirements}{%
\subsubsection{About this format and the R Journal
requirements}\label{about-this-format-and-the-r-journal-requirements}}

\texttt{rticles::rjournal\_article} will help you build the correct
files requirements:

\begin{itemize}
\tightlist
\item
  A R file will be generated automatically using \texttt{knitr::purl} -
  see \url{https://bookdown.org/yihui/rmarkdown-cookbook/purl.html} for
  more information.
\item
  A tex file will be generated from this Rmd file and correctly included
  in \texttt{RJwapper.tex} as expected to build \texttt{RJwrapper.pdf}.
\item
  All figure files will be kept in the default rmarkdown
  \texttt{*\_files} folder. This happens because
  \texttt{keep\_tex\ =\ TRUE} by default in
  \texttt{rticles::rjournal\_article}
\item
  Only the bib filename is to modifed. An example bib file is included
  in the template (\texttt{RJreferences.bib}) and you will have to name
  your bib file as the tex, R, and pdf files.
\end{itemize}

\bibliography{biokNN.bib}

\address{%
Maximiliano Cubillos\\
Aarhus University\\%
\\
%
%
%
\\\href{mailto:mcub@econ.au.dk}{\nolinkurl{mcub@econ.au.dk}}
}

\address{%
Jesper N. Wulff\\
Aarhus University\\%
\\
%
%
%
\\\href{mailto:jwulff@econ.au.dk}{\nolinkurl{jwulff@econ.au.dk}}
}

\address{%
Sanne Wøhlk\\
Aarhus University\\%
\\
%
%
%
\\\href{mailto:sanw@econ.au.dk}{\nolinkurl{sanw@econ.au.dk}}
}
